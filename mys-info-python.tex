\documentclass{article}
%
%\usepackage{trace}
%
\usepackage{luatex}
\ifluatex \else\usepackage{pdfprivacy}\usepackage{fontspec}\fi
%
\setlength{\parindent}{0pt}
%
\def\esphor{\hspace{1cm}}
%
\def\packtested{mys-info-python}
\usepackage{\packtested}
\usepackage{pgffor}
%
%####################################################
%++++++++++++++++++++++++++++++++++++++++++++++++++++
%
%  Titre du chapitre
                     \title{\packtested}
                     \author{moi}
%
%++++++++++++++++++++++++++++++++++++++++++++++++++++
%####################################################
%
\begin{sympycode}
from sympy.stats import DiscreteUniform, sample
x = Symbol('x')
a = DiscreteUniform('a', range(-10, 11))
b = DiscreteUniform('b', range(-10, 11))
c = DiscreteUniform('c', range(-10, 11))
def randquad():
      return Eq(sample(a)*x**2 + sample(b)*x + sample(c))
\end{sympycode}
\newcommand\randquad{\sympy{randquad()}}
%
\begin{document}
%
\maketitle{}
\section{A Faire}
Montrer l'indentation avec les espaces rendus visibles en début de ligne

Un environnement qui montre le code à gauche et le résultat (obtenu avec python) à droite

Meilleures couleurs
\section{Utilisation de pythontex}
\subsection{importing python files}
%https://tex.stackexchange.com/questions/129657/pythontex-importing-python-files
\begin{pyconsole}
import os
os.getcwd()
import sayhi
\end{pyconsole}

\begin{pyconcode}
import os
import sys
sys.path.append(os.getcwd())
\end{pyconcode}

\begin{pyconsole}
import sayhi
sayhi.hi()
\end{pyconsole}


\subsection{tikz \texttt{foreach} to iterate over a python list via pythontex}
% https://tex.stackexchange.com/questions/469263/tikz-foreach-to-iterate-over-a-python-list-via-pythontex
\begin{pycode}
import sys
some_letters = ['A', 'B', 'C']
\end{pycode}

normal for-loop:\\
\foreach \letter in {a, b, c} {letter=\letter\\}

for-loop with \texttt{pythontex}:\\
Version 1:\\
\pyc{print("\\foreach \\letter in {"+','.join(some_letters)+"} {letter=\\letter\\\\}",file=sys.stdout, flush=True)}

\noindent
Version 2.2:\\
\py{"\\foreach \\letter in {"+','.join(some_letters)+"} {letter=\\letter\\\\}"}
%
%
%
\subsection{sympy}
$\randquad$\quad $\randquad$\quad $\randquad$

\section{listing d'un code}
\subsection{listings}
\begin{python}[frame=single]{python}
def f(n):
  if n == 1:
    return 1
  else:
    return f(n-1)
    
print(f(4))
\end{python}
%
\subsection{pygments}
\begin{pygments}[frame=single, indent=L]{python}
def f(n):
  if n == 1:
    return 1
  else:
    return f(n-1)
    
print(f(4))
\end{pygments}
%
\begin{pygments}[frame=single]{python}
def f(n):
  if n == 1:
    return 1
  else:
    return f(n-1)
    
print(f(4))
\end{pygments}
%
%
%
\section{Importation d'un fichier .py}
\subsection{listings}

\pythonexternal[language=Python, label={lst:label32}, showstringspaces=false, extendedchars=true,keepspaces=true, tabsize=4, morekeywords={models, lambda, forms}, commentstyle={\rmfamily\catcode`\$=11}, columns=flexible,texcl,showspaces=false, captionpos=b,caption=Example of using the segmentation script with the help of the PypePad command and a Pypes object.]{./PythonCodePart.py}

\subsection{minted}
\begin{minted}{Python}
#!/usr/bin/env python
import sys
import math
import string
from vmtk import pypes
from vmtk import vmtkscripts

"""triple quoted strings"""

seedPointCoordinates = "399 268 364"
targetPointCoordinates = [" 367 264 361 " , "344 287 336 ",
 " 336 289 330 " , " 315 287 330 " , " 294 278 325 " , " 269 254 321 "]

for x in range (0,6,1):
    myInitializationArguments = "vmtkimageinitialization -ifile <PATH>" 
    \XA_001.dcm -interactive 0 -method fastmarching
    -upperthreshold 11000  -lowerthreshold 2000 -sourcepoints " 
    +seedPointsOne+ 
    " -targetpoints "
    +targetPointsOne[x]+
    " -olevelsetsfile <PATH>\initialLevelSets"+str(x+1)+".vti"

    myPype = pypes.PypeRun(myinitializationArguments)

    myArgumentsForMarchingCubes = "vmtkmarchingcubes –ifile <PATH>\initialLevelSets" 
    +str(x+1)+ ".vti   -ofile <PATH>\initialLevelSets"
    +str(x+1)+ "mc.vtp"
    myPype = pypes.PypeRun(myArgumentsForMarchingCubes)
\end{minted}
\end{document}