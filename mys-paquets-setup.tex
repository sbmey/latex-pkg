%
%%%%%%%%%%%%%%%%%%%%%%%%%%%%%%%%%%%%%%%%%%%%%%%%%%
%
%     Version de \@ifpackagewith qui checke aussi les options globales 
%
%%%%%%%%%%%%%%%%%%%%%%%%%%%%%%%%%%%%%%%%%%%%%%%%%%
%
\ExplSyntaxOn
%
\NewDocumentCommand{\packageoptionsTF}{mmmm}%
    {\stanton_package_options:nnTF { #1 } { #2 } { #3 } { #4 }}%
%
%
\cs_new_protected:Nn \stanton_package_options:nnTF
    {%
    \clist_map_inline:nn { #2 }%
        {%
        \clist_if_in:cnTF { opt@#1.sty }%
            { ##1 }%
            { #3 }% it's a local option
            {%
            \clist_if_in:cnTF { @classoptionslist } { ##1 }%
                { #3 }% it's a global option
                { #4 }%
            }%
        }%
    }%
%
\ExplSyntaxOff
%
%%%%%%%%%%%%%%%%%%%%%%%%%%%%%%%%%%%%%%%%%%%%%%%%%%
%
%     babel    
%
%%%%%%%%%%%%%%%%%%%%%%%%%%%%%%%%%%%%%%%%%%%%%%%%%%
%
%\packageoptionsTF{babel}{french}%
%    {%
%    \DecimalMathComma
%    \frenchsetup
%        {%
%        StandardLists=true						% n°4 	Si true : annulation des actions de frenchb sur les listes (cf n°1, n°2 et n°3). Utile en cas de conflit avec une autre extension ou classe
%%        ,ItemLabels={\small \textbullet}			% n°3	Marqueurs des items des listes itemize
%        ,ItemLabeli={\small \textbullet}			%
%        ,ItemLabelii=\ding{43}					%
%        ,ItemLabeliii=\textendash					%
%        ,ItemLabeliv={\footnotesize \textbullet}	% \ding{43}% load pifont
%        ,IndentFirst=false							% Indentation(retrait) du 1er paragraphe des sections de \parindent
%        ,PartNameFull=false						%\part donne alors "partie I" sinon on obtient "première partie"
%        ,INGuillSpace=true						%
%        }%
%    \listindentFB=0pt%
%    \@ifclassloaded{cours}
%        {%
%        \AtEndPreamble{%
%            \addto\captionsfrench{%
%                \expandafter\renewcommand\expandafter\frenchpartname\expandafter{\@cours@partie@nom}%
%                \expandafter\renewcommand\expandafter\frenchchaptername\expandafter{\@cours@chapitre@nom}%
%                }%
%            }%
%        }%
%        {}%
%    \AtEndPreamble{\addto\extrasfrench{\sisetup{locale = FR}}}%
%    }%
%    {}%
%
%
%
%\ifluatex
%    \@ifpackageloaded{babel}{}{\def\og{<<}\def\fg{>>}}%
%\fi
%
%%%%%%%%%%%%%%%%%%%%%%%%%%%%%%%%%%%%%%%%%%%%%%%%%%
%
%     couleurs    
%
%%%%%%%%%%%%%%%%%%%%%%%%%%%%%%%%%%%%%%%%%%%%%%%%%%
%
\definecolor{almond}{rgb}{0.94, 0.87, 0.8}%
%
%%%%%%%%%%%%%%%%%%%%%%%%%%%%%%%%%%%%%%%%%%%%%%%%%%
%
%     siunitx    
%
%%%%%%%%%%%%%%%%%%%%%%%%%%%%%%%%%%%%%%%%%%%%%%%%%%
%
\sisetup{%
    locale=FR 										%
%    ,load=abbr										%
    ,inter-unit-product = \ensuremath{{}\cdot{}}	% ou seulement {.} ?
    ,output-decimal-marker={,}						%
    }%
%
%%%%%%%%%%%%%%%%%%%%%%%%%%%%%%%%%%%%%%%%%%%%%%%%%%
%
%     enumitem    
%
%%%%%%%%%%%%%%%%%%%%%%%%%%%%%%%%%%%%%%%%%%%%%%%%%%
%
\newlist{partie}{enumerate}{1}%
%
%
%
\setlist[partie,1]%
    {% general list settings (enumitem's itemize, enumerate, and description)
    nosep,							%
    align=left,						% left-aligned enumerate
    topsep=0mm,					% space before enumerate
    label={\bfseries Partie~\Alph*.},	%
    leftmargin=0pt,					% no left indent
    labelsep=0pt,					% align all item bodies vertically
    labelwidth=0pt,					%
    parsep=0em,					%
    itemsep=1mm					% space between enumerate items
%  labelindent=0pt,					%
    }%
%
%
%
\setlist[enumerate,itemize,description]%
    {%
    nosep			%
    ,align=left		% left-aligned enumerate
    ,topsep=0mm	% space before enumerate
    ,leftmargin=*	% no left indent
    ,itemsep=0mm	% space between enumerate items
    ,noitemsep		%
%  labelsep=0.5em,% align all item bodies vertically
%  labelwidth=*,%
%  labelindent=0pt,%
%  listparindent=0em%
    }%
%
%	Counter format
%
\renewcommand{\theenumi}{\arabic{enumi}}%
\renewcommand{\theenumii}{\alph{enumii}}%
\renewcommand{\theenumiii}{\roman{enumiii}}%
%
%	Counter label
%
\renewcommand{\labelenumi}{\theenumi.}%
\renewcommand{\labelenumii}{\theenumii.}%
\renewcommand{\labelenumiii}{\theenumiii.}%
\renewcommand{\labelitemi}{ \textbullet}%
\renewcommand{\labelitemii}{\textopenbullet}%{$\circ$}%
\renewcommand{\labelitemiii}{\ding{213}}%
\renewcommand{\labelitemiv}{\small\textbullet}%
%\renewcommand{\labelitemi}{\small$\bullet$}%
%\renewcommand{\labelitemii}{\ding{43}}%{\small \textbullet}
%\renewcommand{\labelitemiii}{$\ast$}%
%\renewcommand{\labelitemiii}{\footnotesize \textbullet}%
%\renewcommand{\labelitemiv}{{\textbullet}}%
%
% Style d'énumération
% À utiliser ainsi \begin{enumerate}[label=\circled*]
\def\circled#1{\expandafter\@circled\expandafter{\csname c@#1\endcsname}}
\def\@circled#1{\ding{\@arabic{\numexpr#1+171}}}
\AddEnumerateCounter*{\circled}{\@circled}{18}
%
%%%%%%%%%%%%%%%%%%%%%%%%%%%%%%%%%%%%%%%%%%%%%%%%%%
%
%     tasks    
%
%%%%%%%%%%%%%%%%%%%%%%%%%%%%%%%%%%%%%%%%%%%%%%%%%%
%
\settasks{%
    label=\arabic*.			%
    ,item-indent = 1.2em 	%
    ,label-width =1.2em 		%
    ,label-offset = 0em 		%
    ,after-item-skip = -1ex 	%
    ,column-sep=2.3ex 		%
    ,before-skip=-1.5ex 		%
    ,after-skip=-1.5ex 		%
    }%
%
%
%
\settasks{%
    label=\arabic* 					%
    ,item-indent=1.1em 				%
    ,label-offset=0em 				%
    ,column-sep=2em 				%
    ,before-skip =-\parskip-2mm		%
    ,after-skip = -\parskip-1ex 		%
    ,after-item-skip = -\parskip-1ex	%
    }%
%
%
%
\settasks{%
    label=\arabic*. 					%
    ,item-indent=1.1em 				%
    ,label-offset=0em 				%
    ,column-sep=1em 				%
    ,before-skip =-\parskip-2mm 		%
    ,after-skip = -\parskip-1ex		%
    ,after-item-skip = -\parskip-1ex	%
    }%
%
%
%
\settasks{%
    label=\arabic*. 					%
    ,item-indent=1.1em 				%
    ,label-offset=0em 				%
    ,column-sep=1em				%
    ,before-skip =-\parskip-1.75mm 	%
    ,after-skip = -\parskip-1ex 		%
    ,after-item-skip = -\parskip-1ex 	%
    }%
%
%-------------------------------------------------------------------------------------------------------------------------
%
% enumerate et itemize horizontaux (avec tasks)
%
%-------------------------------------------------------------------------------------------------------------------------
%
\NewTasksEnvironment[%
    label=\arabic*.				%
    ,label-format=\bfseries,                %
    ,item-indent=0.778em+0.333em 	%1em+0.3333
    ,label-width=0.778em 				%1em
    ,label-offset=0.333em 			%
    ,label-align=left 					%
    ,column-sep=1em 				%
    ,before-skip =-\topsep 			% espace vertical avant la liste
    ,after-skip =-\topsep 				%-\partopsep-1ex,% espace vertical après la liste
    ,after-item-skip = -\parskip-0.5ex 	%1ex    
    ]{enumerate--}[\item]%
%
\NewTasksEnvironment[%
    label=\arabic*.					%
%    ,label-format=\bfseries,                %
    ,item-indent=1.25em 	%1em+0.3333
    ,label-width=1.25em 				%1em
    ,label-offset=0em 			%
    ,label-align=left 					%
    ,after-item-skip =0ex 	%1ex    
    ]{enumerate-}[\item]%
%
%
%
\NewTasksEnvironment[%
    label=\labelitemi 					%
    ,item-indent=0.778em +0.333em		% 0.6666em
    ,label-width=0.778em 				% 0.6666em
    ,label-offset=0.333em 					% 0em
    ,label-align=left 						%
    ,column-sep=1em 					%
%    before-skip =-\parskip-1.75mm,%
%    after-skip = -\parskip-1ex,
%    after-item-skip = -\parskip-1ex
    ,before-skip =-\topsep-0.3333em 	%
    ,after-skip = -\topsep-0.3333em 		%-\partopsep,%%
    ,after-item-skip =0pt 				%-\topsep%    
    ]{itemize-}[\item]%
%
%\AddToHook{env/tasks/before}{%
%    \ifdim\parskip=0pt\else\vspace{-6pt}\fi}
%
%%%%%%%%%%%%%%%%%%%%%%%%%%%%%%%%%%%%%%%%%%%%%%%%%%
%
%     geometry    
%
%%%%%%%%%%%%%%%%%%%%%%%%%%%%%%%%%%%%%%%%%%%%%%%%%%
%
\geometry% a4= 210mm x 297mm
    {%
    driver=none 		 %
    ,top=24mm         %22mm	%26mm
    ,bottom=20mm	%37mm	%28.7mm,%
    ,inner=21mm      %20mm	%21mm,%
    ,outer=21mm      %20mm	%31.5mm%
    ,a4paper			%
%  ,showframe%
    }%
%
%
%
%
\newcommand*{\demipage}{\geometry{paperheight=148.5mm,bottom=15mm}}% a4= 210mm x 297mm
%
%
%
\newcommand*{\pagepleine}{%
    \pagestyle{empty}%
    \newgeometry{%210*297
        ,top=\baselineskip 	%26.2
        ,bottom=0mm			%37mm%28.7mm,%
        ,inner=8mm			%21mm,%
        ,outer=8mm			%31.5mm%
        ,a4paper				%
        ,ignoreheadfoot		%
    }%
}%
%
%%%%%%%%%%%%%%%%%%%%%%%%%%%%%%%%%%%%%%%%%%%%%%%%%%
%
%     tabularx    
%
%%%%%%%%%%%%%%%%%%%%%%%%%%%%%%%%%%%%%%%%%%%%%%%%%%
%
\newcolumntype{Y}{>{\centering\arraybackslash}X}%
\newcolumntype{V}{>{\centering\arraybackslash}c}%
%
%%%%%%%%%%%%%%%%%%%%%%%%%%%%%%%%%%%%%%%%%%%%%%%%%%
%
%     forest    
%
%%%%%%%%%%%%%%%%%%%%%%%%%%%%%%%%%%%%%%%%%%%%%%%%%%
%
\AtBeginDocument{%
\@ifpackageloaded{forest}%
    {%
    \forestset{%
        gappy/.style={%
            before typesetting nodes={%
                insert after={[\dots, no edge]}%
                }%
            }%
        %
        %
        %
        ,align at top/.style={/tikz/baseline={($(current bounding box.north)-(0,\ht\strutbox)$)}}%
        ,begin draw/.code={\begin{tikzpicture}[baseline={($(current bounding box.north)-(0,\ht\strutbox)$)}]}%
        %
        %
        %
        ,nice empty node/.style={%
            for tree={calign=fixed edge angles}%
            ,delay={where content={}{shape=coordinate,for siblings={anchor=center}}{}}%
            },%
        }%
    \def\getfirst#1;#2\endget{#1}%
    \def\getsecond#1;#2\endget{#2}%
    }%
    {}%
}%
%
%
%
%%%%%%%%%%%%%%%%%%%%%%%%%%%%%%%%%%%%%%%%%%%%%%%%%%
%
%     tikz    
%
%%%%%%%%%%%%%%%%%%%%%%%%%%%%%%%%%%%%%%%%%%%%%%%%%%
%
\usetikzlibrary{math,arrows.meta,tikzmark,patterns,fit,plotmarks,bending}%
%
%
%
%####################################################
%####################################################
%
%  Dimensions (longueur et hauteur) de la bounding box d'une tikzpicture
%
%####################################################
%####################################################
%
%
%
% Trouve les dimensions de la bounding box d'une tikzpicture
%
\newcommand{\pgfsize}[2]{%
% #1 = macro storing width of the bounding box of a tikz picture
% #2 = macro storing height
%    \pgfextractx%
%        {\@tempdima}%
%        {\pgfpointdiff%
%            {\pgfpointanchor{current bounding box}{west}}%
%            {\pgfpointanchor{current bounding box}{east}}%
%        }%
%    \global#1=\@tempdima%
%    \pgfextracty%
%        {\@tempdima}%
%        {\pgfpointdiff%
%            {\pgfpointanchor{current bounding box}{south}}%
%            {\pgfpointanchor{current bounding box}{north}}%
%        }%
%    \global#2=\@tempdima%
    \pgfpointdiff%
        {\pgfpointanchor{current bounding box}{south west}}%
        {\pgfpointanchor{current bounding box}{north east}}%
    \pgfgetlastxy{#1}{#2}%
    \global\let#1#1
    \global\let#2#2
%\show#1\show#2
}%
%
% Trouve les dimensions de la bounding box d'une tikzpicture et définit le path de cette bb
%
\newlength{\mys@utilstikz@bb@longueur}%
\newlength{\mys@utilstikz@bb@hauteur}%
%
\def\mys@boundingboxdimensions{%
% UTILE ? FONCTIONNEL ?
    \pgfsize{\mys@utilstikz@bb@longueur}{\mys@utilstikz@bb@hauteur}%
    \path[use as bounding box]  (0,0) rectangle (\mys@utilstikz@bb@longueur,\mys@utilstikz@bb@hauteur);%
}%
%
% Style (trouve les dimensions de la bb)
%
\pgfkeys{%
    /tikz/boundingboxdimensions/.style=%
        {/tikz/execute at end picture={\pgfsize{\mys@utilstikz@bb@longueur}{\mys@utilstikz@bb@hauteur}}}%
}%
%
%
%
%####################################################
%####################################################
%
%  Imprimer le contenu d'un style (ou d'une clé ?) dans le log
%
%####################################################
%####################################################
%
%
%
%Paul Gaborit : https://tex.stackexchange.com/questions/406975/how-to-see-content-of-a-tikz-style/413527#413527
\pgfkeys{%
    /handlers/.typeout code/.code={%
        \pgfkeysgetvalue{\pgfkeyscurrentpath/.@cmd}{\pgfkeysshower}%
        \typeout{--- \pgfkeyscurrentpath\space---}%
        \typeout{\meaning\pgfkeysshower}%
        }%
    }%
% usage \pgfkeys{/path/to/style/mon style/.typeout code} puis aller regarder le log
%
%
%
%####################################################
%####################################################
%
%  Sauvegarde les coordonnées d'un point (node?) dans deux macros
%
%####################################################
%####################################################
%
%
%
\def\extractcoord#1#2#3{%
    \path let \p1=(#3) in \pgfextra{%
        \pgfmathsetmacro#1{(\x{1}/\pgf@xx)/\sbxunit}%
        \pgfmathsetmacro#2{(\y{1}/\pgf@yy)/\sbyunit}%
        \xdef#1{#1}\xdef#2{#2}%
        };%
    }%
%
%
%
%####################################################
%####################################################
%
%  Extraction des coordonnées d'un node
%
%####################################################
%####################################################
%
%
%
\newcommand{\Getxycoords}[3]{%
% #1: node name
% #2: output macro name: x coordinate
% #3: output macro name: y coordinate
    \pgfplotsextra{%
    % using `\pgfplotspointgetcoordinates' stores the (axis)
    % coordinates in `data point' which then can be called by
    % `\pgfkeysvalueof' or `\pgfkeysgetvalue'
        \pgfplotspointgetcoordinates{(#1)}%
        % `\global' (a TeX macro and not a TikZ/PGFPlots one) allows to
        % store the values globally
        \global\pgfkeysgetvalue{/data point/x}{#2}%
        \global\pgfkeysgetvalue{/data point/y}{#3}%
        }%
    }
%
%
%
%####################################################
%####################################################
%
%  Baseline en haut du graphique
%
%####################################################
%####################################################
%
%
%
\pgfkeys{tikz/topitem/.style={baseline={($(current bounding box.north)-(0,\ht\strutbox)$)}}}

\pgfkeys{tikz/align at top/.style={baseline={($(current bounding box.north)-(0,\ht\strutbox)$)}}}

\pgfkeys{tikz/align at center/.style={baseline={($(current bounding box.center)-(0,\ht\strutbox)$)}}}

%\pgfkeys{tikz/align at center/.style={baseline={($(current bounding box.center)-(0,\ht\strutbox)$)}}}
%
%
%
%####################################################
%####################################################
%
%  Le texte suit la courbe
%
%####################################################
%####################################################
%
%
%
% \usetikzlibrary {decorations.text}
\tikzset{text deco/.style={postaction={decorate, decoration={text along path, #1}}}}%
%
\pgfkeys{tikz/texte suit courbe/.style={postaction={decorate, decoration={text along path, #1}}}}%
%
%
%
%####################################################
%####################################################
%
%  Label entouré pour enumitem     
%
%####################################################
%####################################################
%
%
%
\newcommand*\circledlabel[1]{%
    \begin{tikzpicture}[baseline=(C.base)]%
    \node[draw,circle,inner sep=1pt,minimum size=3ex](C) {#1};%
    \end{tikzpicture}%
    }%
%
%
%
%####################################################
%####################################################
%
%  Points     
%
%####################################################
%####################################################
%
%
%
\tikzset{point/.style={
    thick,
    fill=black,
    circle,%
    inner sep=0pt,
    minimum width=3.5pt,
    minimum height=3.5pt,
    }%
}
%
%
%
%####################################################
%####################################################
%
%  Vecteurs     
%
%####################################################
%####################################################
%
%
%
\pgfkeys{%
    tikz/vecteur/.style args={#1}%
        {#1},%
    vecteur/.default= {-{Stealth[]}}%
    }%
%
%
%
%####################################################
%####################################################
%
%  Marques     
%
%####################################################
%####################################################
%
%
%
\pgfkeys{%
    tikz/parallel/.style args={#1,#2}%
        {% 
        decoration={%
            markings,% switch on markings
            mark= at position #1 with
                {\foreach \n in{1,...,#2}{\pgfmathparse{2*\n-2}\draw[xshift=\pgfmathresult pt] (-3pt,-3pt)--  (0pt,3pt);}},% 
            },%
        postaction={decorate}%, opacity=0
        },%
    parallel/.default= {0.5,2}%
    }%
%
%
%
\tikzset{%
    || mark/.style={%
        postaction=decorate,%
        decoration={%
            markings,%
            mark=at position #1 with {%
                \draw[line cap=round,mark segment] (-1pt,-2pt) -- (-1pt,2pt);%
                \draw[line cap=round,mark segment] (1pt,-2pt) -- (1pt,2pt);%
                }%
            }%
        },%
%
    ||| mark/.style={%
        postaction=decorate,%
        decoration={%
            markings,%
            mark=at position #1 with {%
                \draw[line cap=round,mark segment] (-2pt,-2pt) -- (-2pt,2pt);%
                \draw[line cap=round,mark segment] (0pt,-2pt) -- (0pt,2pt);%
                \draw[line cap=round,mark segment] (2pt,-2pt) -- (2pt,2pt);%
                }%
            }%
        },%
%
    mark segment/.style={thick}%
    }%
%
%
%
%####################################################
%####################################################
%
%  Droite graduée     
%
%####################################################
%####################################################
%
%
%
\pgfkeys{%
    /mys/droitegraduee/.cd,%
    begin space/.store in = \mys@drtgrad@begin@space,%
    xmin/.store in = \mys@drtgrad@x@min,%
    xmax/.store in = \mys@drtgrad@x@max,%
    end space/.store in = \mys@drtgrad@end@space,%
    grad/.store in = \mys@drtgrad@grad,%
    unit/.store in = \mys@drtgrad@unit,%
    subgrad/.store in = \mys@drtgrad@subgrad,%
    grad width/.store in = \mys@drtgrad@grad@linewidth,%
    subgrad width/.store in = \mys@drtgrad@subgrad@linewidth,%
    line width/.store in = \mys@drtgrad@line@linewidth,%
    }%
%
%
%
\newcommand*\droitegraduee[3][]{%
    \begingroup%
    \pgfkeys{%
        /mys/droitegraduee/.cd,%
        begin space=0.25cm,%
        end space=0.5cm,%
        grad=1,%
        subgrad=0,%
        subgrad width=0.5pt,%
        line width=0.75pt,%
        grad width=0.75pt,%
        #1,%
        xmin=#2,%
        xmax=#3%
    }%
    %
    \pgfmathgreater{\mys@drtgrad@subgrad}{0}%
    \ifnum\pgfmathresult=1\relax%
        \foreach \i in {0,\mys@drtgrad@subgrad,...,\mys@drtgrad@x@max}{\draw[line width=\mys@drtgrad@subgrad@linewidth] (\i,-2pt)--(\i,2pt);}%
        \foreach \i in {0,-\mys@drtgrad@subgrad,...,\mys@drtgrad@x@min}{\draw[line width=\mys@drtgrad@subgrad@linewidth] (\i,-2pt)--(\i,2pt);}%
    \fi
    %
    \draw[line width=\mys@drtgrad@line@linewidth,-{Latex[]}]([xshift=-\mys@drtgrad@begin@space]\mys@drtgrad@x@min,0)--([xshift=\mys@drtgrad@end@space]\mys@drtgrad@x@max,0);%
    %
    \foreach \i in {0,-\mys@drtgrad@grad,...,\mys@drtgrad@x@min}{%
        \draw[line width=\mys@drtgrad@grad@linewidth] (\i,-2pt)--(\i,2pt);
        \node[below,anchor=north] at (\i,-2pt) {$\pgfmathprintnumber[zerofill=false]{\i}$};    
    }%
    \foreach \i in {0,\mys@drtgrad@grad,...,\mys@drtgrad@x@max}{%
        \draw[line width=\mys@drtgrad@grad@linewidth] (\i,-2pt)--(\i,2pt);
        \node[below,anchor=north] at (\i,-2pt) {$\pgfmathprintnumber[zerofill=false]{\i}$};    
    }%
    \endgroup
    }%%
%
\definecolor{saumon1}{RGB}{239,126,147}
\definecolor{mauve1}{RGB}{225,175,209}
\definecolor{mauve2}{RGB}{210,146,179}
\definecolor{turquoise1}{RGB}{0,206,209}
\definecolor{bleuvert}{RGB}{23,222,209}
%
%%%%%%%%%%%%%%%%%%%%%%%%%%%%%%%%%%%%%%%%%%%%%%%%%%
%
%     pgfplots    
%
%%%%%%%%%%%%%%%%%%%%%%%%%%%%%%%%%%%%%%%%%%%%%%%%%%
%
\usepgfplotslibrary{groupplots,fillbetween}%
\pgfplotsset{compat=newest}
%
%
%
%####################################################
%####################################################
%
%  Tangentes
%
%####################################################
%####################################################
%
%
%
%\usepgfplotslibrary{fillbetween}
%
\def\parsenode[#1]#2\pgf@nil{%
    \tikzset{label node/.style={#1}}
    \def\nodetext{#2}
    }%
%
%
%
\tikzset{%
    add node at x/.style 2 args={%
        name path global=plot line,%
        /pgfplots/execute at end plot visualization/.append={%
            \begingroup
            \@ifnextchar[{\parsenode}{\parsenode[]}#2\pgf@nil
            \path [name path global = position line #1-1]%
                ({axis cs:#1,0}|-{rel axis cs:0,0}) -- ({axis cs:#1,0}|-{rel axis cs:0,1});%
            \path [xshift=1pt, name path global = position line #1-2]%
                ({axis cs:#1,0}|-{rel axis cs:0,0}) -- ({axis cs:#1,0}|-{rel axis cs:0,1});%
            \path [%
                name intersections={%
                    of={plot line and position line #1-1},%
                    name=left intersection%
                },%
                name intersections={%
                    of={plot line and position line #1-2},%
                    name=right intersection%
                },%
                label node/.append style={pos=1}%
            ] (left intersection-1) -- (right intersection-1) node [label node]{\nodetext};
            \endgroup
        }%
    }%
}%
%
%
%
\pgfplotsset{%
    tangent/.style={%
        add node at x={#1}{%
            [%
            sloped,%
            append after command={(\tikzlastnode.west) edge [thick,black] (\tikzlastnode.east)},%
            minimum width=0.2\textwidth%
            ]%
        }%
    }%
}%
%
% Usage
%
% \addplot[color=black,smooth, tangent/.list={2,3}] coordinates {(-5,-5) (-3,-4)  (-2,-1)  (0,0)  (1,2)  (2,3)  (3,3)  (4,1)};%
%
%
%
%%%%%%%%%%%%%%%%%%%%%%%%%%%%%%%%%%%%%%%%%%%%%%%%%%
%
%     tkz-base    
%
%%%%%%%%%%%%%%%%%%%%%%%%%%%%%%%%%%%%%%%%%%%%%%%%%%
%
\pgfkeys{/tkzdrawY/.cd,  noticks = false}% CORRECTION à tkz-obj-axes.tex de tkz-base
%
%%%%%%%%%%%%%%%%%%%%%%%%%%%%%%%%%%%%%%%%%%%%%%%%%%
%
%     multicols    
%
%%%%%%%%%%%%%%%%%%%%%%%%%%%%%%%%%%%%%%%%%%%%%%%%%%
%
\setlength\multicolsep{\topsep}%
\setlength\multicolsep{0pt}%
%
%%%%%%%%%%%%%%%%%%%%%%%%%%%%%%%%%%%%%%%%%%%%%%%%%%
%
%     comment    
%
%%%%%%%%%%%%%%%%%%%%%%%%%%%%%%%%%%%%%%%%%%%%%%%%%%
%
%\ExplSyntaxOn
%%
%\NewDocumentCommand{\exclure}{m}%
%    {\clist_map_inline:nn {#1}{\excludecomment{##1}}}
%%
%\NewDocumentCommand{\inclure}{m}%
%    {\clist_map_inline:nn {#1}{\includecomment{##1}}}
%%
%\ExplSyntaxOff
%
%%%%%%%%%%%%%%%%%%%%%%%%%%%%%%%%%%%%%%%%%%%%%%%%%%
%
%     pgfpages    
%
%%%%%%%%%%%%%%%%%%%%%%%%%%%%%%%%%%%%%%%%%%%%%%%%%%
%
\def\deuxpagessurune{%
    \geometry{paperheight=148.5mm,bottom=15mm}% a4= 210mm x 297mm
    \pgfpagesuselayout{2 on 1}[a4paper,border shrink=-2mm]%
    \pgfpageslogicalpageoptions{1}{border code=\pgfstroke}%
  }%
%
%
%
\def\npagessurune#1{%
    \geometry{paperheight=297mm/#1,bottom=15mm}% a4= 210mm x 297mm
    \pgfpagesuselayout{#1 on 1}[a4paper,border shrink=-2mm]%
%    \pgfpageslogicalpageoptions{1}{border code=\pgfstroke}%
    \pgfpageslogicalpageoptions{2}{border code=\pgfstroke}%
  }%
%
%
%
\pgfpagesdeclarelayout{3 on 1}%
{%
  \edef\pgfpageoptionheight{\the\paperwidth}% landscaped by default
  \edef\pgfpageoptionwidth{\the\paperheight}%
  \def\pgfpageoptionborder{0pt}%
}%
{%
  \pgfpagesphysicalpageoptions
  {%
    logical pages=3,%
    physical height=\pgfpageoptionheight,%
    physical width=\pgfpageoptionwidth,%
  }%
    \pgfpageslogicalpageoptions{1}%
    {%
      border shrink=\pgfpageoptionborder,%
      resized width=\pgfphysicalwidth,%
      resized height=.333333\pgfphysicalheight,%
      center=\pgfpoint{.5\pgfphysicalwidth}{.833333\pgfphysicalheight}%
    }%
    \pgfpageslogicalpageoptions{2}%
    {%
      border shrink=\pgfpageoptionborder,%
      resized width=\pgfphysicalwidth,%
      resized height=.333333\pgfphysicalheight,%
      center=\pgfpoint{.5\pgfphysicalwidth}{.5\pgfphysicalheight}%
    }%
    \pgfpageslogicalpageoptions{3}%
    {%
      border shrink=\pgfpageoptionborder,%
      resized width=\pgfphysicalwidth,%
      resized height=.333333\pgfphysicalheight,%
      center=\pgfpoint{.5\pgfphysicalwidth}{.166667\pgfphysicalheight}%
    }%
}%
%
%%%%%%%%%%%%%%%%%%%%%%%%%%%%%%%%%%%%%%%%%%%%%%%%%%
%
%     tcolorbox    
%
%%%%%%%%%%%%%%%%%%%%%%%%%%%%%%%%%%%%%%%%%%%%%%%%%%
%
% Fixing the tables with tcolorbox
%\renewcommand*\l@tcolorbox{\@dottedtocline{1}{1.5em}{3.0em}}%
%
%%%%%%%%%%%%%%%%%%%%%%%%%%%%%%%%%%%%%%%%%%%%%%%%%%
%
%     numprint    
%
%%%%%%%%%%%%%%%%%%%%%%%%%%%%%%%%%%%%%%%%%%%%%%%%%%
%
%
\AtBeginDocument{%
\npstylefrench
    \npdecimalsign{,}%
%    \npfourdigitnosep
%    \npthousandsep{}%
%    \npaddmissingzero
}%
%
%%%%%%%%%%%%%%%%%%%%%%%%%%%%%%%%%%%%%%%%%%%%%%%%%%
%
%     tabu    
%
%%%%%%%%%%%%%%%%%%%%%%%%%%%%%%%%%%%%%%%%%%%%%%%%%%
%
\NewDocumentCommand\mysrowheight{O{-#2}m}{\rule[#1]{0pt}{#2}}%
%
%
%%%%%%%%%%%%%%%%%%%%%%%%%%%%%%%%%%%%%%%%%%%%%%%%%%
%
%     tabularray    
%
%%%%%%%%%%%%%%%%%%%%%%%%%%%%%%%%%%%%%%%%%%%%%%%%%%
%
\UseTblrLibrary{diagbox}
%
%%%%%%%%%%%%%%%%%%%%%%%%%%%%%%%%%%%%%%%%%%%%%%%%%%
%
%     fontspec    
%
%%%%%%%%%%%%%%%%%%%%%%%%%%%%%%%%%%%%%%%%%%%%%%%%%%
%
\@ifpackageloaded{fontspec}%
    {%
    \defaultfontfeatures{Ligatures=TeX}%
    \@ifpackageloaded{libertinus}%
        {
        \setmainfont[%
            Path=/usr/share/fonts/libertinus/,%
            Extension=.otf,%
            UprightFont={*-Regular},%
            BoldFont={*-Bold},%
            ItalicFont={*-Italic},%
            BoldItalicFont={*-BoldItalic}%
            ]{LibertinusSerif}%
        }
        {% TeXoMaker semble ne pas pouvoir charger le paquet libertinus        
        }
%    \setmainfont{TeXGyreTermesX}[
%         SmallCapsFeatures={Letters=SmallCaps}, % <=== See Section 4.1 of Fontspec documentation.  
%         UprightFeatures = {SmallCapsFont=texgyreheros-regular.otf}, 
%         BoldFeatures = {SmallCapsFont=texgyreheros-bold.otf},
%        ]
%    \setmainfont{texgyretermes}[
%Extension = {.otf},
%UprightFont = {*-regular}, 
%ItalicFont = {*-italic},
%BoldFont = {*-bold}, 
%BoldItalicFont = {*-bolditalic},
%UprightFeatures = {SmallCapsFont=texgyreheros-regular.otf, SmallCapsFeatures={Letters=SmallCaps, Color=992211}},
%BoldFeatures    = {SmallCapsFont=/Users/John/Library/Fonts/FiraGO-Heavy.otf,   SmallCapsFeatures={Letters=SmallCaps, Color=112299}, Color = FF4422},
%]
%    \setmainfont{Cambria}
    }%
    {}%
%
%%%%%%%%%%%%%%%%%%%%%%%%%%%%%%%%%%%%%%%%%%%%%%%%%%
%
%     unicode-math    
%
%%%%%%%%%%%%%%%%%%%%%%%%%%%%%%%%%%%%%%%%%%%%%%%%%%
%
%https://tex.stackexchange.com/questions/425098/which-opentype-math-fonts-are-available/425099#425099
%    \setmathfont{Libertinus Math}%
%    %\setmathfont[range={}]{Libertinus Math}% Voir https://tex.stackexchange.com/questions/421846/prevent-setmathfont-influence-on-overline
%    %\setmathfont{XITS Math}
%\setmathfont{Cambria Math}
%
\@ifpackageloaded{unicode-math}%
    {%
    \@ifpackageloaded{libertinus}%
        {
        \setmathfont{Latin Modern Math}[version=LM]%
        \setboldmathrm{Libertinus Serif Bold}[Ligatures=Common, ItalicFont=Libertinus Serif Bold Italic]%
        }
        {% TeXoMaker semble ne pas pouvoir charger le paquet libertinus        
        }
%    \setmathfont[]{Libertinus Math}
%%    \setmathfont{STIX2Math}
%%    \setmathfont{Libertinus Math}[version=Libertinus]
%%    \setmathfont{Neo Euler}[version=Euler]
%%    \setmathfont{TeX Gyre DejaVu Math}[version=DejaVu]
%%    \setmathfont{Stix Math}[version=Stix]
%%    \setmathfont{XITS Math}[StylisticSet=1,version=XITS1]
%%    \setmathfont{Stix Two Math}[version=Stix2] % see http://stixfonts.org/
%%    \setmathfont{Cambria Math}[version=Cambria]
%%    \setmathfont{Asana Math}[version=Asana]
%%    \setmathfont{Asana Math}[Alternate,version=AsanaAlt]
%%    \setmathfont{TeX Gyre Pagella Math}[version=Pagella]
%%    \setmathfont{TeX Gyre Termes Math}[version=Termes]
%%    \setmathfont{TeX Gyre DejaVu Math}[version=DejaVu] 
%%    \setmathfont[range=\mathup]  {Lato Light}
%%    \setmathfont[range=\mathbfup]{Lato Bold}
%%    \setmathfont[range=\mathbfit]{Lato Bold Italic}
%%    \setmathfont[range=\mathit]  {Lato Light Italic}
    }%
    {}%
%
%%%%%%%%%%%%%%%%%%%%%%%%%%%%%%%%%%%%%%%%%%%%%%%%%%
%
%     Fontes (kernel : pas de paquets à charger)    
%
%%%%%%%%%%%%%%%%%%%%%%%%%%%%%%%%%%%%%%%%%%%%%%%%%%
%
%Latex kernel, no package needed :
%     \DeclareSymbolFontAlphabet{\mathrm}    {operators}%  normal upright Roman font 
%     \DeclareSymbolFontAlphabet{\mathnormal}{letters}%  normal math italic font (\mathnormal{a}same as      $a$)
%     \DeclareSymbolFontAlphabet{\mathcal}   {symbols}% special calligraphic font for uppercase letters only
%     \DeclareMathAlphabet      {\mathbf}{OT1}{cmr}{bx}{n}% upright Roman boldface letters
%     \DeclareMathAlphabet      {\mathsf}{OT1}{cmss}{m}{n}% upright sans serif letters
%     \DeclareMathAlphabet      {\mathit}{OT1}{cmr}{m}{it}% text italic letters: $different\ne\mathit{different}$
%     \DeclareMathAlphabet      {\mathtt}{OT1}{cmtt}{m}{n}%     upright letters from the typewriter type font
%     \DeclareSymbolFont{operators}   {OT1}{cmr} {m}{n}
%     \DeclareSymbolFont{letters}     {OML}{cmm} {m}{it}
%     \DeclareSymbolFont{symbols}     {OMS}{cmsy}{m}{n}
%
%%%%%%%%%%%%%%%%%%%%%%%%%%%%%%%%%%%%%%%%%%%%%%%%%%
%
%     Fontes (kernel : pas de paquets à charger)    
%
%%%%%%%%%%%%%%%%%%%%%%%%%%%%%%%%%%%%%%%%%%%%%%%%%%
%
\ifluatex
% A FAIRE :
% PROBLEME DE L'ESPACEMENT VERTICAL POUR LES NUMERATEURS ET LES DENOMINATEURS
    \begin{luacode*}
    local function patch_libertinus_frac(fontdata)
    if fontdata.psname == "Libertinus Math" then
        local mc = fontdata.MathConstants
        mc.FractionNumeratorShiftUp = 0.4 * mc.FractionNumeratorShiftUp
        mc.FractionDenominatorShiftDown = 0.4 * mc.FractionDenominatorShiftDown
        end
    end
    luatexbase.add_to_callback("luaotfload.patch_font", patch_libertinus_frac, "libertinus_frac")
    \end{luacode*}
\fi
%
%%%%%%%%%%%%%%%%%%%%%%%%%%%%%%%%%%%%%%%%%%%%%%%%%%
%
%     polyglossia    
%
%%%%%%%%%%%%%%%%%%%%%%%%%%%%%%%%%%%%%%%%%%%%%%%%%%
%
    \@ifpackageloaded{polyglossia}%
        {%
        \setdefaultlanguage{french}%
%        \apptocmd{\french@punctuation}%
%        {%
%        LuaTeXinterchartoks\french@punctguillstart 255={\nobreakspace\xpg@nospace}%
%        LuaTeXinterchartoks 255 \french@punctguillend={\xpg@unskip\nobreakspace}%  
%        }{}{}%
        \def\og{<<\,}\def\fg{\,>>}%
        \gappto\captionsfrench{%
            \renewcommand{\partname}{Partie}%
            \renewcommand{\thepart}{\Alph{part}}%
            \renewcommand\contentsname{Table~des~matières}%
            \renewcommand\chaptername{Chapitre}%
%    \@ifundefined{subchaptername}{\newcommand{\subchaptername}{Sous-chapitre}}{\renewcommand{\subchaptername}{Sous-chapitre}}% UTILE ?
            \renewcommand\appendixname{Annexe}%
            \renewcommand\listtablename{Liste~des~tableaux}%
            }%
%        \MakeOuterQuote{''}
        }{}%
%
%%%%%%%%%%%%%%%%%%%%%%%%%%%%%%%%%%%%%%%%%%%%%%%%%%
%
%     Fontes TI    
%
%%%%%%%%%%%%%%%%%%%%%%%%%%%%%%%%%%%%%%%%%%%%%%%%%%
%
\ifluatex
    \newfontfamily{\TIfont}{ti-83-plus.ttf}%
%    \newfontfamily{\TIfont}{ti-84.ttf}
    \newfontfamily{\TIfontkey}{ti-83premium-keys.ttf}%
%    \newfontfamily{\TIfontkey}{TI84PlusCEKeys.ttf}
%    \newfontfamily{\TIfontkey}{ti-76fr-keys.ttf}
%    \newfontfamily{\TIfontkey}{ti-84-keys.ttf}
%    \newfontfamily{\TIfontkey}{ti-89-keys.ttf}
%    \newfontfamily{\TIfontkey}{ti-92-keys.ttf}
%    \newfontfamily{\TIfontkey}{ti-baii-keys.ttf}
    \newfontfamily{\Casiofont}{Casio-Graph-35plus.ttf}%
    \newfontfamily{\Casiofontkey}{Casio-ClassPad2-CP401-Key.ttf}%
%    \newfontfamily{\Casiofontkey}{Casio-ClassPad2-CP402-Key.ttf}
%    \newfontfamily{\Casiofontkey}{Casio-ClassPad2-CP403-Key.ttf}
%    \newfontfamily{\Casiofontkey}{Casio-ClassPad2-CP404-Key.ttf}
%    \newfontfamily{\Casiofontkey}{Casio-ClassPad2-CP405-Key.ttf}
%    \newfontfamily{\Casiofontkey}{Casio-ClassPad2-CP406-Key.ttf}
%    \newfontfamily{\Casiofontkey}{Casio-ClassPad2-CP407-Key.ttf}
%    \newfontfamily{\Casiofontkey}{Casio-ClassPad2-CP408-Key.ttf}
%    \newfontfamily{\Casiofontkey}{Casio-ClassPad330-CP301-Key.ttf}
%    \newfontfamily{\Casiofontkey}{Casio-ClassPad330-CP302-Key.ttf}
%    \newfontfamily{\Casiofontkey}{Casio-ClassPad330-CP303-Key.ttf}
%    \newfontfamily{\Casiofontkey}{Casio-ClassPad330-CP304-Key.ttf}
%    \newfontfamily{\Casiofontkey}{Casio-ClassPad330-CP305-Key.ttf}
%    \newfontfamily{\Casiofontkey}{Casio-ClassPad330-CP306-Key.ttf}
%    \newfontfamily{\Casiofontkey}{Casio-ClassWiz-Fr-Key.ttf}
%    \newfontfamily{\Casiofontkey}{Casio-ClassWiz-Key.ttf}
%    \newfontfamily{\Casiofontkey}{Casio-GraphicSeries-CFX01-Key.ttf}
%    \newfontfamily{\Casiofontkey}{Casio-GraphicSeries-CFX02-Key.ttf}
%    \newfontfamily{\Casiofontkey}{Casio-GraphicSeries-CFX03-Key.ttf}
%    \newfontfamily{\Casiofontkey}{Casio-GraphicSeries-CFX04-Key.ttf}
%    \newfontfamily{\Casiofontkey}{Casio-MS01-Key.ttf}
%    \newfontfamily{\Casiofontkey}{Casio-ScientificStandardModel-ES03-Key.ttf}
%    \newfontfamily{\Casiofontkey}{Casio-ScientificStandardModel-ES04-Key.ttf}
\else
    \newcommand\TIfont[1]{\usefont{T1}{ti-83-plus.ttf}{m}{n}#1}%
    \newcommand\TIfontkey[1]{\usefont{T1}{ti-83premium-keys.ttf}{m}{n}#1}%
    \newcommand\Casiofont[1]{\usefont{T1}{Casio-Graph-35plus.ttf}{m}{n}#1}%
    \newcommand\Casiofontkey[1]{\usefont{T1}{Casio-ClassPad2-CP401-Key.ttf}{m}{n}#1}%
\fi
%
%
%
\DeclareTextFontCommand{\tifont}{\TIfont}%
\DeclareTextFontCommand{\tifontkey}{\TIfontkey}%
\DeclareTextFontCommand{\casioscreen}{\Casiofont}%
\DeclareTextFontCommand{\casiokey}{\CasioCPikey}%
%
%
%
%----------%----------%----------%----------%----------%----------%----------%----------%----------%----------%----------%
%
%\newfontfamily{\TIfontbis}{ti-83premium-keys}[% <- Use the name on your system
%    Path=/usr/share/fonts/TTF/,% the right path, if needed
%    Extension=.ttf,% fix for your system
%    ]
%\DeclareTextFontCommand{\tifontbis}{\TIfontbis}
%
%
%
\newcommand{\TI}[1]{\tifontkey{#1}}%
%
%
%
%%% --- Pour afficher un tableau des codes des caractères ( à placer dans le corpsd'un docyument)
%\begin{multicols}{4}
%\TIfontbis
%
%\count255=0
%\loop\ifnum\count255<10000
%\iffontchar\font\count255
%  \hbox{\makebox[2.5em][r]{\normalfont\the\count255\quad}\TIfontbis\char\count255}
%\fi
%\advance\count255 1
%\repeat
%
%\end{multicols}
%
%
%
\newcommand{\faketikey}[1]{%
    \tikz[baseline={(0,-0.73ex)}]%
        \node[draw,rounded corners=0.65pt,inner sep=1pt,line width=0.6pt]%
            {\resizebox{!}{1.2ex}{\fontspec{Cantarell-Regular.otf}#1}};%
    }%
%
%
%
\ExplSyntaxOn
\NewDocumentCommand{\TIs}{m}{%
    \group_begin:%
    \TIfontkey{%
        \clist_map_inline:nn { #1 }%
            {\symbol { \str_case:nV { ##1 } \c_tikeys_translation_tl }}%
        }%
    \group_end:%
    }%
%
\tl_const:Nn \c_tikeys_translation_tl{%
    {to}{33}%
    {scatter}{34}%
    {grey}{35}%
    {black}{36}%
    {barplot}{38}%
    {space}{39}%
    {graph}{40}%
    {regression}{41}%
    {boxplotdots}{42}%
    {boxplot}{43}%
    {statplot}{44}%
    {tablset}{45}%
    {format}{46}%
    {calc}{47}%
    {table}{48}%
    {>}{49}%
    {>>}{50}%
    {>>|}{51}%
    {e}{52}%
    % ...
    {atan}{65}%
    {pi}{66}%
    {sqrt}{67}%
    {EE}{68}%
    {lbrace}{69}%
    {rbrace}{70}%
    {10^x}{71}%
    {lbrack}{72}%
    {rbrack}{73}%
    % ...
    {+}{195}%
    {-}{185}%
    {0}{202}%
    {1}{192}%
    {2}{193}%
    {3}{194}%
    {4}{182}%
    {5}{183}%
    {6}{184}%
    {7}{172}%
    {8}{170}%
    {9}{174}%
    % ...
    {(}{163}%
    {)}{164}%
    {(-)}{204}%
    {/}{165}%
    {div}{165}%
    {*}{175}%
    {math}{187}%
    {mode}{122}%
%    {trig}{118}% PAS DISPONIBLE
    {arcsin}{63}%
    {arccos}{64}%
    {desc}{8224}%
    {^}{8250}%
    {entrer}{205}%
    {trace}{114}%
    {.}{203}%
    {²}{161}%
    {x2}{161}%
    {2nd}{121}%
    {graph}{115}%
    {prgm}{188}%
    {dessin}{60}%
    {f(x)}{111}%
}%
%
% Pour utiliser \faketikey lorsque c'est nécessaire (trig, racine cubique , ...)
%    \__ti_select_key:n{
%        \tl_set:
%        \tl_case:NnTF{}{}
%    }
%
\ExplSyntaxOff
%
%  ticalc
%
\colorlet{shadecolor}{green!35!black!25}%
%
\newcommand\ticalc[1][16]{%
    \ifhmode\par\fi
    \begingroup
        \parindent0pt%
        \def\nbchar{#1}%
        \catcode`\;12%
        \catcode`\:12%
        \catcode`\!12%
        \catcode`\?12%
        \catcode`\^^M12%
        \catcode`\ 12%
        \catcode`\112
        \ticalci
        }%
%
%
%
\newcommand\ticalci[1]{%
        \normalsize\TIfont\FrameSep=5pt%
        \hsize\dimexpr0.8em*\nbchar+2\FrameSep\relax
        \def\>{%
            \hbox{%
                -\llap{\lower0.122em\hbox{`}\llap{\lower0.366em\hbox{\reflectbox{`}}\kern0.122em }}%
                }%
            }%
        \def\!{\hbox{=\llap{/}}}%
        \def\1{\hbox{\bfseries\scriptsize{1}}}%
        \csname shaded*\endcsname\leavevmode\ticalcii#1\relax\csname endshaded*\endcsname
    \endgroup
    }%
%
%
%
\newcommand\ticalcminibox[1][16]{%
    \ifhmode\par\fi
    \begingroup
        \parindent0pt \fboxrule=1pt \fboxsep=4pt%
        \def\nbchar{#1}%
        \catcode`\;12%
        \catcode`\112%
        \catcode`\:12%
        \catcode`\!12%
        \catcode`\?12%
        \catcode`\^^M12%
        \catcode`\ 12%
        \normalsize\TIfont
        \def\>{%
            \hbox{%
                -\llap{%
                    \lower0.122em%
                    \hbox{`}%
                    \llap{\lower0.366em\hbox{\reflectbox{`}}\kern0.122em }%
                    \kern-0.122em%
                    }%
                }%
            }%
        \def\!{\hbox{=\llap{/}}}%
        \def\1{\hbox{\bfseries\scriptsize{1}}}%
        \ticalcminiboxi
    }%
%
%
%
\newcommand\ticalcminiboxi[1]{%
    \fcolorbox{black}%
        {green!30!black!25}%
        {\vbox{\hsize\dimexpr0.8em*\nbchar\relax\leavevmode\ticalcii#1\relax}}%
    \par
    \endgroup
    }%
%
\begingroup\catcode`\^^M=12\relax%
\gdef\ticalcii#1{%
    \unless\ifx\relax#1%
    \ifx^^M#1\par\leavevmode\else\hbox to0.8em{\hss#1\hss}\hskip0pt \fi\expandafter\ticalcii\fi%
    }%
\endgroup%
%
%%%%%%%%%%%%%%%%%%%%%%%%%%%%%%%%%%%%%%%%%%%%%%%%%%
%
%     hyperref    
%
%%%%%%%%%%%%%%%%%%%%%%%%%%%%%%%%%%%%%%%%%%%%%%%%%%
%
\@ifclassloaded{cours}%
    {%
     \hypersetup{%
         pdftitle={},		% Cours de mathématiques
         pdfsubject={}	% Un cours de math
         }%
    }%
    {}%
%
%
%
\@ifclassloaded{eval}%
    {%
     \hypersetup{%
         pdftitle={},		% Évaluation de mathématiques
         pdfsubject={}	% Une évaluation de math
         }%
    }%
    {}%
%
%
%
\hypersetup{%
    colorlinks=true				% color the words instead of use a colored box
    ,allcolors=.					% . est une couleur spéciale dans xcolor : c'est la couleur courante. Comme ça les la couleur des liens est celle du texte autour : on voit pas les liens !
    ,urlcolor=blue				% ,citecolor=black, % \cite{}
    ,linktoc=all					%
    ,pdfencoding=auto			% or option "unicode"
    ,psdextra						% additional math macros are supported in bookmarks
    ,pdfborder={0 0 0}			% for removing borders around links
    ,breaklinks=true				%
    ,bookmarksnumbered=true	%
    ,pdfcreator = {}				% LaTeX with hyperref package
    ,pdfpagelabels=true			%
    ,pdfpagemode=UseOutlines	% None, UseThumbs, UseOutlines, FullScreen
%    ,plainpages=false,pdfpagelabels			%
%    ,letterpaper=true
%    ,plainpages=false
%% plainpages boolean true
%% Forces page anchors to be named by the arabic form of the page number,
%% rather than the formatted form.
%% breaklinks boolean false
%% Allows link text to break across lines; since this cannot be accommodated in
%% PDF, it is only set true by default if the pdftex driver is used. This makes
%% links on multiple lines into different PDF links to the same target.
%%    ,pagebackref=true
%% Adds ?backlink? text to the end of each item in the bibliography, as a list
%% of section numbers. This can only work properly if there is a blank line
%% after each \bibitem.
%% bookmarksnumbered boolean false
%% If Acrobat bookmarks are requested, include section numbers.
%    ,bookmarksopen=false%
%% bookmarksopen boolean false
%% If Acrobat bookmarks are requested, show them with all the subtrees expanded.
%%pdfsubject={},
}%
%
%%%%%%%%%%%%%%%%%%%%%%%%%%%%%%%%%%%%%%%%%%%%%%%%%%
%
%     hyperxmp    
%
%%%%%%%%%%%%%%%%%%%%%%%%%%%%%%%%%%%%%%%%%%%%%%%%%%
%
% Métadatas : autres que l'heure et la date 
%
\hypersetup{%
%    pdfaconformance={},			%
%    pdfapart={},					%
    baseurl={},						% http://pas.durl.org
    pdfauthor={Mon professeur},	%
    pdfauthortitle={},				%
    pdfcaptionwriter={},				%
    pdfcontactaddress={},			%
    pdfcontactcity={},				%
    pdfcontactcountry={},			%
    pdfcontactemail={},				%
    pdfcontactphone={},			%
    pdfcontactpostcode={},			%
    pdfcontactregion={},			%
    pdfcontacturl={},				% http://urldemonlycee.none/
    pdfcopyright={Ne pas partager, et encore moins avec la boîte à bac à côté du lycée},	%
    pdfkeywords={},				%
    pdflang={},						% fr
    pdflicenseurl={},				% http://creativecommons.org/licenses/by-nc-nd/3.0/    ---->    URL pointant vers la certification du respect de la license par le document (... je crois ...)
    pdfmetalang={},				% fr-FR    ---->    langue des metadatas 'titre', 'sujet' et 'copyright'
    pdfproducer={},					% luaLaTeX
    pdfsource={},					% \jobname.tex
    pdftype={},						% Fiches d'activités, de cours, d'exercices ou de devoirs
    pdftitle={},						%
    }%
%
% Métadatas : date et heure
%
%---- Choix date
%---- si nécessaire on fait précéder le chiffre du mois ou du jour d'un 0
%
\ifnum\month<10\edef\moiscourant{0\the\month}\else\edef\moiscourant{\the\month}\fi %
\ifnum\day<10\edef\jourcourant{0\the\day}\else\edef\jourcourant{\the\day}\fi %
%
\edef\datechoisie{\the\year\moiscourant\jourcourant}%
%
%---- Choix heure
%
\def\heurechoisie{120000}% 12h00'00''
%
%---- Réglage
%
\hypersetup{%
    pdfmetadate={D:\datechoisie\heurechoisie+02'00'},		% date des metadatas du document
    pdfdate={D:\datechoisie\heurechoisie+02'00'},				% date du document
    pdfcreationdate={D:\datechoisie\heurechoisie+02'00'},		%
    pdfmoddate={D:\datechoisie\heurechoisie+02'00'},			%
    }%
%
%---- Définition des dates dans les métadonnées du pdf
%
\def\metadatadate#1#2#3{%
    \def\datedebut{#3#2#11000}%
    \AtBeginDocument{%
        \hypersetup{%
            pdfmetadate={D:\datedebut+02'00'},		% date des metadatas du document
            pdfdate={D:\datedebut+02'00'},				% date du document
            pdfcreationdate={D:\datedebut+02'00'},		%
            pdfmoddate={D:\datedebut+02'00'}			%
        }%
    }%
}%
%
%%%%%%%%%%%%%%%%%%%%%%%%%%%%%%%%%%%%%%%%%%%%%%%%%%
%
%     bookmark    
%
%%%%%%%%%%%%%%%%%%%%%%%%%%%%%%%%%%%%%%%%%%%%%%%%%%
%
\pdfbookmark[3]{Bookmarktitle}{section}%
%
%%%%%%%%%%%%%%%%%%%%%%%%%%%%%%%%%%%%%%%%%%%%%%%%%%
%
%     code divers
%
%%%%%%%%%%%%%%%%%%%%%%%%%%%%%%%%%%%%%%%%%%%%%%%%%%
%
\newcommand{\cuthere}{%
    \smallskip%
    \par\noindent%
    \makebox[\linewidth]{\raisebox{0pt}{\makebox[2\paperwidth]{\dotfill}}}%
    \par%
    \smallskip%
    }%
%
%%%%%%%%%%%%%%%%%%%%%%%%%%%%%%%%%%%%%%%%%%%%%%%%%%
%
%     sagetex    
%
%%%%%%%%%%%%%%%%%%%%%%%%%%%%%%%%%%%%%%%%%%%%%%%%%%
%
% https://tex.stackexchange.com/questions/563190/how-to-remove-zero-decimal-in-numbers/563216#563216
%%=============================================================================
%% Paraphernalia:
%%    \UD@firstoftwo, \UD@secondoftwo, \UD@Exchange, \UD@Removespace
%%    \UD@CheckWhetherNull, \UD@CheckWhetherLeadingSpace, \UD@ExtractFirstArg
%%=============================================================================
\newcommand\UD@firstoftwo[2]{#1}%
\newcommand\UD@secondoftwo[2]{#2}%
\newcommand\UD@Exchange[2]{#2#1}%
\@ifdefinable\UD@Removespace{\UD@Exchange{ }{\def\UD@Removespace}{}}%
%%-----------------------------------------------------------------------------
%% Check whether argument is empty:
%%.............................................................................
%% \UD@CheckWhetherNull{<Argument which is to be checked>}%
%%                     {<Tokens to be delivered in case that argument
%%                       which is to be checked is empty>}%
%%                     {<Tokens to be delivered in case that argument
%%                       which is to be checked is not empty>}%
%%
%% The gist of this macro comes from Robert R. Schneck's \ifempty-macro:
%% <https://groups.google.com/forum/#!original/comp.text.tex/kuOEIQIrElc/lUg37FmhA74J>
\newcommand\UD@CheckWhetherNull[1]{%
  \romannumeral0\expandafter\UD@secondoftwo\string{\expandafter
  \UD@secondoftwo\expandafter{\expandafter{\string#1}\expandafter
  \UD@secondoftwo\string}\expandafter\UD@firstoftwo\expandafter{\expandafter
  \UD@secondoftwo\string}\expandafter\expandafter\UD@firstoftwo{ }{}%
  \UD@secondoftwo}{\expandafter\expandafter\UD@firstoftwo{ }{}\UD@firstoftwo}%
}%
%%-----------------------------------------------------------------------------
%% Check whether argument's first token is a catcode-1-character
%%.............................................................................
%% \UD@CheckWhetherBrace{<Argument which is to be checked>}%
%%                      {<Tokens to be delivered in case that argument
%%                        which is to be checked has leading
%%                        catcode-1-token>}%
%%                      {<Tokens to be delivered in case that argument
%%                        which is to be checked has no leading
%%                        catcode-1-token>}%
\newcommand\UD@CheckWhetherBrace[1]{%
  \romannumeral0\expandafter\UD@secondoftwo\expandafter{\expandafter{%
  \string#1,}\expandafter\UD@firstoftwo\expandafter{\expandafter
  \UD@secondoftwo\string}\expandafter\expandafter\UD@firstoftwo{ }{}%
  \UD@firstoftwo}{\expandafter\expandafter\UD@firstoftwo{ }{}\UD@secondoftwo}%
}%
%%-----------------------------------------------------------------------------
%% Check whether brace-balanced argument starts with a space-token
%%.............................................................................
%% \UD@CheckWhetherLeadingSpace{<Argument which is to be checked>}%
%%                             {<Tokens to be delivered in case <argument
%%                               which is to be checked>'s 1st token is a
%%                               space-token>}%
%%                             {<Tokens to be delivered in case <argument
%%                               which is to be checked>'s 1st token is not
%%                               a space-token>}%
\newcommand\UD@CheckWhetherLeadingSpace[1]{%
  \romannumeral0\UD@CheckWhetherNull{#1}%
  {\expandafter\expandafter\UD@firstoftwo{ }{}\UD@secondoftwo}%
  {\expandafter\UD@secondoftwo\string{\UD@CheckWhetherLeadingSpaceB,#1 }{}}%
}%
\newcommand\UD@CheckWhetherLeadingSpaceB{}%
\long\def\UD@CheckWhetherLeadingSpaceB#1 {%
  \expandafter\UD@CheckWhetherNull\expandafter{\UD@firstoftwo{}#1}%
  {\UD@Exchange{\UD@firstoftwo}}{\UD@Exchange{\UD@secondoftwo}}%
  {\UD@Exchange{ }{\expandafter\expandafter\expandafter\expandafter
   \expandafter\expandafter\expandafter}\expandafter\expandafter
   \expandafter}\expandafter\UD@secondoftwo\expandafter{\string}%
}%
%%=============================================================================
%% Extract K-th inner undelimited argument:
%%
%% \UD@ExtractKthArg{<integer K>}{<list of undelimited args>} 
%% 
%% In case there is no K-th argument in <list of indelimited args> : 
%%   Does not deliver any token.
%% In case there is a K-th argument in <list of indelimited args> : 
%%   Does deliver that K-th argument with one level of braces removed.
%%
%% Examples:
%%
%%   \UD@ExtractKthArg{0}{ABCDE} yields: <nothing>
%%
%%   \UD@ExtractKthArg{3}{ABCDE} yields:  C
%%
%%   \UD@ExtractKthArg{3}{AB{CD}E} yields:  CD
%%
%%   \UD@ExtractKthArg{4}{{001}{002}{003}{004}{005}} yields: 004
%%
%%   \UD@ExtractKthArg{6}{{001}{002}{003}} yields: <nothing> 
%% 
%%=============================================================================
\newcommand\UD@ExtractKthArg[1]{%
  \romannumeral0%
  % #1: <integer number K>
  \expandafter\UD@ExtractKthArgCheck
  \expandafter{\romannumeral\number\number#1 000}%
}%
\newcommand\UD@ExtractKthArgCheck[2]{%
  \UD@CheckWhetherNull{#1}{ }{%
    \expandafter\UD@ExtractKthArgLoop\expandafter{\UD@firstoftwo{}#1}{#2}%
  }%
}%
\newcommand\UD@ExtractKthArgLoop[2]{%
  \expandafter\UD@CheckWhetherNull\expandafter{\UD@firstoftwo#2{},}{ }{%
    \UD@CheckWhetherNull{#1}{%
      \UD@ExtractFirstArgLoop{#2UD@SelDOm}%
    }{%
      \expandafter\UD@Exchange\expandafter{\expandafter{\UD@firstoftwo{}#2}}%
      {\expandafter\UD@ExtractKthArgLoop\expandafter{\UD@firstoftwo{}#1}}%
    }%
  }%
}%
\@ifdefinable\UD@RemoveTillUD@SelDOm{%
  \long\def\UD@RemoveTillUD@SelDOm#1#2UD@SelDOm{{#1}}%
}%
\newcommand\UD@ExtractFirstArgLoop[1]{%
  \expandafter\UD@CheckWhetherNull\expandafter{\UD@firstoftwo{}#1}%
  {\UD@firstoftwo{\expandafter}{} \UD@secondoftwo{}#1}%
  {\expandafter\UD@ExtractFirstArgLoop\expandafter{\UD@RemoveTillUD@SelDOm#1}}%
}%
%%=============================================================================
%% Fork if argument, which must be a single token, is
%% 0/1/2/3/4/5/6/7/8/9/+/-/./,/<space token>/<expandable token>/<something else>
%% (total: 17 cases)
%%-----------------------------------------------------------------------------
\@ifdefinable\UD@GobbleToExclam{\long\def\UD@GobbleToExclam#1!{}}%
%%-----------------------------------------------------------------------------
\@ifdefinable\UD@normalizenumberfork{%
   \long\def\UD@normalizenumberfork#1!0!1!2!3!4!5!6!7!8!9!+!-!,!.!#2#3!!!!{#2}%
}%
\newcommand\UD@normalizenumberloopfork[1]{%
  \expandafter\UD@CheckWhetherNull\expandafter{\UD@GobbleToExclam#1!}{%
    \UD@normalizenumberfork
    !#1!1!2!3!4!5!6!7!8!9!+!-!,!.!{1}% <digit> 0_12
    !0!#1!2!3!4!5!6!7!8!9!+!-!,!.!{2}% <digit> 1_12
    !0!1!#1!3!4!5!6!7!8!9!+!-!,!.!{3}% <digit> 2_12
    !0!1!2!#1!4!5!6!7!8!9!+!-!,!.!{4}% <digit> 3_12
    !0!1!2!3!#1!5!6!7!8!9!+!-!,!.!{5}% <digit> 4_12
    !0!1!2!3!4!#1!6!7!8!9!+!-!,!.!{6}% <digit> 5_12
    !0!1!2!3!4!5!#1!7!8!9!+!-!,!.!{7}% <digit> 6_12
    !0!1!2!3!4!5!6!#1!8!9!+!-!,!.!{8}% <digit> 7_12
    !0!1!2!3!4!5!6!7!#1!9!+!-!,!.!{9}% <digit> 8_12
    !0!1!2!3!4!5!6!7!8!#1!+!-!,!.!{10}% <digit> 9_12
    !0!1!2!3!4!5!6!7!8!9!#1!-!,!.!{11}% <plus or minus> +_12
    !0!1!2!3!4!5!6!7!8!9!+!#1!,!.!{12}% <plus or minus> -_12
    !0!1!2!3!4!5!6!7!8!9!+!-!#1!.!{13}% <decimal constant> ,_12
    !0!1!2!3!4!5!6!7!8!9!+!-!,!#1!{14}% <decimal constant> ._12
    !0!1!2!3!4!5!6!7!8!9!+!-!,!.!{%
      \ifcat\noexpand#1 \expandafter\UD@firstoftwo\else\expandafter\UD@secondoftwo\fi
      {15}% <space token> differing from explicit character token of catcode 10 
          % and charcode 32; removable as undelimited argument
      {%
         \expandafter\ifx\noexpand#1#1%
         \expandafter\UD@firstoftwo\else\expandafter\UD@secondoftwo\fi
         {18}% something else which is not allowed
         {17}% expandable token
      }%
    }% 
    !!!!%
  }{18}% Case: #1 contains !_12 , therefore is something else which is not
       % allowed
}%
%%=============================================================================
%% \normalizenumber{<argument>}
%%-----------------------------------------------------------------------------
\newcommand\MysNormalizenumber[1]{%
  \romannumeral0%
  \normalizenumberloop{#1}{}{}{#1}{\UD@firstoftwo}{}{\UD@firstoftwo}{}%
}%
\newif\ifnormalizenumberexpandarg\normalizenumberexpandargfalse
\newcommand\normalizenumberloop[8]{%
  % #1 - argument to iterate
  % #2 - leading zero if found
  % #3 - optional minus sign
  % #4 - argument untouched
  % #5 - decimal separator not/already found - \UD@firstoftwo/\UD@secondoftwo 
  % #6 - zero-decimals collected so far
  % #7 - sign-check on/off - \UD@firstoftwo/\UD@secondoftwo
  % #8 - significant digits collected so far
  \UD@CheckWhetherNull{#1}{%
    \UD@CheckWhetherNull{#8}{\UD@CheckWhetherNull{#2}{ #4}{ #2}}{ #3#8}%
  }{%
    \UD@ExtractKthArg{%
     %-------------------------------------------------------------------------
     % \UD@ExtractKthArg's <integer K>:
     %-------------------------------------------------------------------------
      % Code for calculating \UD@ExtractKthArg's <integer K>
      \UD@CheckWhetherBrace{#1}{%
        18% argument to iterate's 1st token has catcode 1, therefore is not
          % allowed.
      }{%
        \UD@CheckWhetherLeadingSpace{#1}{%
          16% explicit character token of catcode 10 and charcode 32; not
            % removable as undelimited argument
        }{%
          \expandafter\UD@normalizenumberloopfork
          \expandafter{\romannumeral0\UD@ExtractFirstArgLoop{#1UD@SelDOm}}%
        }%
      }%
    }{%
     %-------------------------------------------------------------------------
     % \UD@ExtractKthArg's <list of undelimited args>:
     %-------------------------------------------------------------------------
       % \UD@ExtractKthArg's <list of undelimited args>'s 1st argument:
       %     \UD@ExtractKthArg's 1st argument yields the number 1, thus #1's 
       %     1st token is <digit> 0_12
      {%
        #5{%
          \UD@CheckWhetherNull{#8}{%
            \UD@firstoftwo{%
              \expandafter\normalizenumberloop\expandafter{\UD@firstoftwo{}#1}{0}{#3}{#4}{#5}{}{\UD@secondoftwo}{#8}%
            }%
          }{\UD@Exchange{{#80}}}%
        }{%
          \UD@firstoftwo{%
            \expandafter\normalizenumberloop
            \expandafter{\UD@firstoftwo{}#1}{#2}%
            {#3}{#4}{#5}{#60}{\UD@secondoftwo}{#8}%
          }%
        }%
      }%
       %-----------------------------------------------------------------------
       % \UD@ExtractKthArg's <list of undelimited args>'s 2nd argument:
       %     \UD@ExtractKthArg's 1st argument yields the number 2, thus #1's 
       %     1st token is <digit> 1_12
      {\UD@Exchange{{#8#61}}}%
       %-----------------------------------------------------------------------
       % \UD@ExtractKthArg's <list of undelimited args>'s 3rd argument:
       %     \UD@ExtractKthArg's 1st argument yields the number 3, thus #1's 
       %     1st token is <digit> 2_12
      {\UD@Exchange{{#8#62}}}%
       %-----------------------------------------------------------------------
       % \UD@ExtractKthArg's <list of undelimited args>'s 4th argument:
       %     \UD@ExtractKthArg's 1st argument yields the number 4, thus #1's
       %     1st token is <digit> 3_12
      {\UD@Exchange{{#8#63}}}%
       %-----------------------------------------------------------------------
       % \UD@ExtractKthArg's <list of undelimited args>'s 5th argument:
       %     \UD@ExtractKthArg's 1st argument yields the number 5, thus #1's 
       %     1st token is <digit> 4_12
      {\UD@Exchange{{#8#64}}}%
       %-----------------------------------------------------------------------
       % \UD@ExtractKthArg's <list of undelimited args>'s 6th argument:
       %     \UD@ExtractKthArg's 1st argument yields the number 6, thus #1's
       %     1st token is <digit> 5_12
      {\UD@Exchange{{#8#65}}}%
       %-----------------------------------------------------------------------
       % \UD@ExtractKthArg's <list of undelimited args>'s 7th argument:
       %     \UD@ExtractKthArg's 1st argument yields the number 7, thus #1's
       %     1st token is <digit> 6_12
      {\UD@Exchange{{#8#66}}}%
       %-----------------------------------------------------------------------
       % \UD@ExtractKthArg's <list of undelimited args>'s 8th argument:
       %     \UD@ExtractKthArg's 1st argument yields the number 8, thus #1's
       %     1st token is <digit> 7_12
      {\UD@Exchange{{#8#67}}}%
       %-----------------------------------------------------------------------
       % \UD@ExtractKthArg's <list of undelimited args>'s 9th argument:
       %     \UD@ExtractKthArg's 1st argument yields the number 9, thus #1's
       %     1st token is <digit> 8_12
      {\UD@Exchange{{#8#68}}}%
       %-----------------------------------------------------------------------
       % \UD@ExtractKthArg's <list of undelimited args>'s 10th argument:
       %     \UD@ExtractKthArg's 1st argument yields the number 10, thus #1's
       %     1st token is <digit> 9_12
      {\UD@Exchange{{#8#69}}}%
       %-----------------------------------------------------------------------
       % \UD@ExtractKthArg's <list of undelimited args>'s 11th argument:
       %     \UD@ExtractKthArg's 1st argument yields the number 11, thus #1's
       %     1st token is <plus or minus> +_12
      {%
        \UD@firstoftwo{%
          #7{%
            \expandafter\UD@CheckWhetherNull
            \expandafter{\UD@firstoftwo{}#1}{ #4}{%
              \expandafter\normalizenumberloop
              \expandafter{\UD@firstoftwo{}#1}{#2}{#3}{#4}{#5}{#6}{#7}{#8}%
            }%
          }{ #4}%
        }%
      }%
       %-----------------------------------------------------------------------
       % \UD@ExtractKthArg's <list of undelimited args>'s 12th argument:
       %     \UD@ExtractKthArg's 1st argument yields the number 12, thus #1's
       %     1st token is <plus or minus> -_12
      {%
        \UD@firstoftwo{%
          #7{%
            \expandafter\UD@CheckWhetherNull
            \expandafter{\UD@firstoftwo{}#1}{ #4}{%
              \UD@CheckWhetherNull{#3}{\UD@Exchange{{-}}}{\UD@Exchange{{}}}%
              {\expandafter\normalizenumberloop\expandafter{\UD@firstoftwo{}#1}{#2}}%
              {#4}{#5}{#6}{#7}{#8}%
            }%
          }{ #4}%
        }%
      }%
       %-----------------------------------------------------------------------
       % \UD@ExtractKthArg's <list of undelimited args>'s 13th argument:
       %     \UD@ExtractKthArg's 1st argument yields the number 13, thus #1's
       %     1st token is <decimal constant> ,_12
      {%
        \UD@firstoftwo{%
          #5{%
            \expandafter\UD@CheckWhetherNull
            \expandafter{\UD@firstoftwo{}#1}{ #4}{%
              \UD@CheckWhetherNull{#2#8}{ #4}{%
                \UD@CheckWhetherNull{#8}{\UD@Exchange{{#2}}}{\UD@Exchange{{#8}}}%
                {%
                  \expandafter\normalizenumberloop\expandafter{\UD@firstoftwo{}#1}%
                  {#2}{#3}{#4}{\UD@secondoftwo}{,}{\UD@secondoftwo}%
                }%
              }%
            }%
          }{ #4}%
        }%
      }%
       %-----------------------------------------------------------------------
       % \UD@ExtractKthArg's <list of undelimited args>'s 14th argument:
       %     \UD@ExtractKthArg's 1st argument yields the number 14, thus #1's
       %     1st token is <decimal constant> ._12
      {%
        \UD@firstoftwo{%
          #5{%
            \expandafter\UD@CheckWhetherNull
            \expandafter{\UD@firstoftwo{}#1}{ #4}{%
              \UD@CheckWhetherNull{#2#8}{ #4}{%
                \UD@CheckWhetherNull{#8}{\UD@Exchange{{#2}}}{\UD@Exchange{{#8}}}%
                {%
                  \expandafter\normalizenumberloop\expandafter{\UD@firstoftwo{}#1}%
                  {#2}{#3}{#4}{\UD@secondoftwo}{,}{\UD@secondoftwo}%
                }%
              }%
            }%
          }{ #4}%
        }%
      }%
       %-----------------------------------------------------------------------
       % \UD@ExtractKthArg's <list of undelimited args>'s 15th argument:
       %     \UD@ExtractKthArg's 1st argument yields the number 15, thus #1's
       %     1st token is a <space token> differing from explicit character
       %     token of catcode 10 and charcode 32 and is removable as
       %     undelimited argument
      {%
        \UD@firstoftwo{%
          #7{\UD@firstoftwo}{%
            \expandafter\UD@CheckWhetherNull\expandafter{\UD@firstoftwo{}#1}%
          }%
          {%
            \expandafter\normalizenumberloop
            \expandafter{\UD@firstoftwo{}#1}{#2}{#3}{#4}{#5}{#6}{#7}{#8}%
          }%
          { #4}%
        }%
      }%
       %-----------------------------------------------------------------------
       % \UD@ExtractKthArg's <list of undelimited args>'s 16th argument:
       %     \UD@ExtractKthArg's 1st argument yields the number 16, thus #1's
       %     1st token is a <space token>, more precisely an explicit
       %     character token of catcode 10 and charcode 32 and is not removable
       %     as undelimited argument
      {%
        \UD@firstoftwo{%
          #7{\UD@firstoftwo}{%
            \expandafter\UD@CheckWhetherNull\expandafter{\UD@Removespace#1}%
          }%
          {%
            \expandafter\normalizenumberloop
            \expandafter{\UD@Removespace#1}{#2}{#3}{#4}{#5}{#6}{#7}{#8}%
          }%
          { #4}%
        }%
      }%
       %-----------------------------------------------------------------------
       % \UD@ExtractKthArg's <list of undelimited args>'s 17th argument:
       %     \UD@ExtractKthArg's 1st argument yields the number 17, thus #1's 
       %     1st token is expandable.
      {%
        \UD@firstoftwo{%
          \ifnormalizenumberexpandarg
            \expandafter\UD@firstoftwo\else\expandafter\UD@secondoftwo\fi
          {\expandafter\normalizenumberloop\expandafter{#1}{#2}{#3}{#4}{#5}{#6}{#7}{#8}}%
          { #4}%
        }%
      }%
       %-----------------------------------------------------------------------
       % \UD@ExtractKthArg's <list of undelimited args>'s 18th argument:
       %     \UD@ExtractKthArg's 1st argument yields the number 18, thus #1's
       %     1st token is not allowed with numbers that can be normalized.
      {%
        \UD@firstoftwo{ #4}%
      }%
     %-------------------------------------------------------------------------
     % End of \UD@ExtractKthArg's <list of undelimited args>.
     %-------------------------------------------------------------------------
    }%
    {\expandafter\normalizenumberloop\expandafter{\UD@firstoftwo{}#1}%
                            {#2}{#3}{#4}{#5}{}{\UD@secondoftwo}%
    }%
  }%
}%
%%.............................................................................